\chapter{Локализация людей на плоскости земли} \label{chapt4}

Алгоритм определения положения людей в сцене является отображением из входного изображения в координаты человека на плоскости земли. Для формальной постановки этой задачи необходимо дать определение положению человека на плоскости земли. В качестве положения я рассматриваю проекцию на плоскость земли некоторой фиксированной точки тела человека --- его ``центра''. Важно отметить, что положение центра масс не стоит использовать в качестве такой фиксированной точки, поскольку его положение существенно зависит от позы человека. В качестве ``центра'' тела человека я использовал фиксированную точку внутри его туловища.

\section{Математическая модель наблюдаемых данных}

Я использую модель наблюдаемых данных, описанную в \ref{chap-cam_pose::sec-model}. Используемая модель человека \cite{pishchulin15arxiv} определяет модельную систему координат, центр которой расположен внутри туловища человека. Я использовал центр модельной системы координат модели человека в качестве его ``центра''.

Таким образом, положение человека на плоскости земли я определил как проекцию центра модельной системы координат модели человека на плоскость земли.

\section{Предложенный метод}

\section{Обучение и экспериментальная оценка}
\subsection{Обучение}
\subsection{Экспериментальная оценка на синтетической выборке}
