\chapter{Определение позы человека в видеопоследовательности} \label{chapt6}

В этой главе я рассматриваю задачу определения позы человека в видеопоследовательности. Как описано в обзоре \ref{chapt-related::human_pose_definithion}, в настоящий момент научное сообщество определяет позу человека на изображении как положение его $K$ суставов. Таким образом, позой человека в момент времени $t$ является последовательность точек $P^t \in \left\{ \left\{p_i^t\right\}_{i=1}^K | p_i^t \in \mathbb{R}^2\right\}$ на изображении.

В своей работе я рассматриваю алгоритм определения позы человека в качестве следующего этапа обработки видеоданных после сопровождения. Поэтому моя постановка задачи определения позы в видеопоследовательности имеет следующий вид:
\begin{itemize}
	\item[Вход:] видеопоследовательность $I=\left\{I_t\right\}_{i=1}^N$;
	\item[Выход:] поза человека на каждом кадре $P=\left\{P^t\right\}_{t=1}^{N}$.
\end{itemize}

\section{Математическая модель наблюдаемых данных}

Я рассматриваю определение позы человека в видео как задачу минимизации функции энергии $E(P|I)$. Можно считать, что она задает ненормированную функцию правдоподобия позы в видеопоследовательности в виде $\tilde{p}(P|I) = \exp\left(-E(P|I)\right)$. В дальнейшем для упрощения выкладок я буду неявно предполагать зависимость функции энергии от исходной видеопоследовательности, т.~е. $E(P) = E(P|I)$.

Используемая модель наблюдаемых данных является обобщением модели позы человека на изображении на случай видеопоследовательности. Для этого базовая модель расширяется предположением о зависимости позы человека на разных кадрах. Стандартный подход к расширению модели позы человека на случай видеопоследовательности соответствует следующей функции энергии:
\begin{equation}
	\begin{aligned}
		E(P) &= \sum_{t=1}^T E_I(P^t) + \sum_{t=1}^{T-1}E_T(P^{t+1}, P^{t}),
	\end{aligned}
\end{equation}
где $E_I(P^t)$ "--- модель позы человека на кадре \eqref{eq::frame}, а $E_T(P^{t+1}, P^{t})$ "--- модель изменения позы между кадрами. Такую модель можно рассматривать как марковскую цепь первого порядка, где состояние в каждый момент времени является многомерной величиной и описывает позу человека.

Она состоит из двух частей: 1) базовая модели позы человека на изображении $E_I(P^t)$ и 2) модель движения её суставов $E_T(P^{t+1}, P^{t})$.

\subsection{Модель позы человека на изображении}

Согласно обзору существующих методов в научных работах представлены две основные модели позы человека: модель из набора деформируемых частей и регрессионная модель позы.

Несмотря на то, что вторая модель на момент написания данной диссертации позволяет добиться лучших результатов определения позы на изображении, её обобщение на случай видеопоследовательности затруднено. Построенное отображение входного изображения $I_t$ на позу человека $P^t$ на нем не позволяет использовать априорные сведения, полученные на соседних кадрах.

Поэтому в качестве базовой модели позы человека на изображении была выбрана модель из набора деформируемых частей. Соответствующая ей графическая модель, определяет ненормированную функцию правдоподобия $\tilde{p}(P^t) = \exp(-E_I(P^t))$, а следовательно позволяет интегрировать её как часть большей графической модели, используя информацию с предыдущих кадров для построения априорных ограничений.

Модель из набора частей описывается с помощью функции энергии $E_I(P_t)$, минимум которой определяется в качестве позы человека на изображении $I$:
\begin{equation}
	E_I(P^t) = \sum_{i\in V}{\phi_i(p_i^t, s^t)} + \sum_{\left(i,j\right)\in E}{\psi_{(i,j)}^s(p_i^t, p_j^t, s^t)}
	\label{eq::frame}
\end{equation}
Параметр $s^t$ задаёт масштаб тела человека. Хотя формально в определение позы человека этот параметр не входит, для удобства обозначений в дальнейшем я буду предполагать, что параметр $P^t$ его содержит.

\subsection{Модель движения}


Наиболее простым способом задания модели изменения позы является предположение о независимости движения суставов позы
\begin{equation}
	\Psi(P^{t+1}, P^{t}) = \sum_{i\in V}\psi_i^t(p_i^{t+1}, p_i^t, s^t)
\end{equation}

Так как модели движения разных суставов схожи, то достаточно рассмотреть её лишь для одного из них. Для упрощения обозначений в данном подразделе я опускаю индекс рассматриваемого сустава. Например, $p^t$ используется для обозначения состояния рассматриваемого сустава на кадре $I_t$.

Такое расширение модели позы человека на изображении на случай видеопоследовательности использовался также в предыдущих работах. Например, в работе \cite{park2011n} предлагалась модель движения, предполагающее слабое изменение позы человека между кадрами:
\begin{equation*}
	\psi_i^t(p^{t+1}, p^t, s^t) = -\frac{1} {2 {s^t}^2}(p^{t+1} - p^t)^T \Sigma_p^{-1} (p^{t+1} - p^t)
\end{equation*}
Таким образом, оптимальное значение такой модели движения достигается при постоянстве позы человека в видео. Изменение позы при движении оказывается <<допустимым шумом>>.

В своей работе я расширил эту модель движения. Я использовал линейную динамическую систему для описания движения суставов тела человека:
\begin{equation}
	\begin{aligned}
		\psi(h^{t+1}, h^{t}, s^{t}) &=
			-\frac{1}{2 {s^t}^2} (h^{t+1} - A h^t)^T \Sigma_p^{-1} (h^{t+1} - A h^t) \\
	\end{aligned}
	\label{eq::temp}
\end{equation}
Неизвестное состояние $h^t$ включает в себя положение и характеристики движения сустава. В этой работе я рассматривал линейную модель движения суставов, т.~e. скрытое состояние для каждого сустава описывается его положением на кадре и мгновенной скоростью движения:
\begin{equation}
	\begin{aligned}
		h^t &= \left[p^t, v^t\right], \quad v^t \in \mathbb{R}^2\\
		A &=
			\begin{bmatrix}
			1 & 0 & 1 & 0 \\
			0 & 1 & 0 & 1 \\
			0 & 0 & 1 & 0 \\
			0 & 0 & 0 & 1 \\
			\end{bmatrix} \\
	\end{aligned}
\end{equation}

Допустимое отклонение от линейной модели движения задается с помощью матрицы симметричной положительно определенной матрицы $\Sigma_p \in \mathbb{S_+}$.  Хотя предлагаемый метод не зависит конкретного выбора матрицы $\Sigma_p$, для уменьшения количества параметров я рассматривал только диагональные матрицы следующего вида:
\begin{equation}
	\begin{aligned}
		\Sigma_p &= \left[
			\begin{array}{c|c}
			\Sigma_p^p & \Theta \\ \hline
			\Theta     & \Sigma_p^v
			\end{array}
			\right] \\
		\Sigma_p^p &= \alpha_p^{-1} I_{2\times2} \\
		\Sigma_p^v &= \alpha_v^{-1} I_{2\times2} \\
		\alpha_p &> 0, \alpha_v > 0,
	\end{aligned}
\end{equation}
Матрица $\Sigma_p^p$ описывает допустимое отклонение положения сустава $p^{t+1}$ от его линейного предсказания $p^t + v_t$, a $\Sigma_p^v$ "--- допустимое изменение скорости сустава между кадрами.
Таким образом, в предложенной модели позы человека появляются дополнительные скрытые параметры скорости движения суставов.

Использование только фактора \eqref{eq::temp} допускает неправдоподобно большое значение скорости движения сустава. Чтобы этого избежать, был добавлен фактор, задающий априорное распределение на скорость движения суставов в начальный момент времени:
\begin{equation}
	\begin{aligned}
		\psi(V_1) &= -\frac{V_0^T \Sigma_p^{v_0} V_0}{2 S_0^2} \\
		\Sigma_p^{v_0} &= \alpha_{v_0}^{-1} I_{2\times2}
	\end{aligned}
\end{equation}

Аналогичную модель изменения можно предложить для параметра размера тела человека $S_i$. Однако, в данной работе я ограничился рассмотрением более простой модели изменения этого параметра:
\begin{equation}
	\psi^s(S_{i-1}, S_i) = -\frac{1}{2} \left(\frac{S_i - S_{i-1}}{S_{i-1}\sigma_s}\right)^2
\end{equation}

\subsection{Частные случаи}

Предложенная модель имеет два интересных частных случая, рассматриваемых в предыдущих работах.

Можно показать, что предложенная модель является обобщением модели \cite{park2011n}. Действительно, при значениях $\alpha_v \rightarrow \inf, \alpha_{v_0} \rightarrow \inf$, функции их энергии совпадают.

С другой стороны если $\alpha_v \rightarrow 0, \alpha_{v_0} \rightarrow 0$, то функция энергии не содержит ограничений на значения и изменение скорости суставов на разных кадрах. Таким образом модель сводится к независимому определению позы человека на каждом кадре.

\section{Метод оптимизации}

\subsection{Анализ модели} P.S. показать, что глобальная оптимизация невозможна

Каждой функции энергии, описывающей модель позы человека, соответствует марковская сеть. От её свойств зависит сложность алгоритмов поиска оптимального значения модели. В общем случае сложность оптимизации не меньше количества допустимых состояний минимальной группы вершин, разделяющих графическую модель на две части. Если графическая модель является деревом, то сложность алгоритма квадратично зависит от количества допустимых состояний для каждой вершины. Примером такой графической модели является модель позы человека на изображении при условии известного параметра масштаба. Специальный вид парных потенциалов, позволил сократить сложность вывода в графической модели до линейной зависимости от количества допустимых положений суставов на изображении.

К сожалению, при расширении этой модели, в марковской сети появляются циклы, а следовательно сложность поиска оптимального решения существенно возрастает. На рисунке \ref{} представлена графическая модель, соответствующая модели \cite{park2011n}. Если размер последовательности превосходит количество суставов в позе человека, то поза вершины позы человека на кадре образуют минимальное количество вершин, разделяющих графическую модель на две части. Можно заметить, что на высоком уровне эта модель представляет марковскую цепь, но каждая её вершина описывает позу человека на одном кадре. Если на каждом кадре имеется $M$ возможных положений сустава, и поза состоит из $K$ суставов, то сложность алгоритма распространения доверия равна $\mathcal{O}(M^{K}T)$. Здесь существенно использован факт, что парный потенциал $\psi$ является квадратичной формой, и для вычисления сообщений может быть использован метод дистантного преобразования. Таким образом, сложность поиска оптимального множества поз линейно зависит от количества возможных поз человека на изображении. Так как значение параметра $M$ может превосходить $10^4$, а количество суставов исчисляться десятками, то алгоритм поиска точного решения оказывается не применим на практике. Уменьшив количество допустимых поз на кадре изображения, авторы \cite{park2011n} смогли построить приближенное решение задачи.

Предложенная в данной работе модель является расширением \cite{park2011n} и содержит её в качестве частного случая. Также скрытые параметры позы дополнительно включают скорость суставов, а значит содержат больше состояний. Также в своей модели я считаю скорость движения суставов непрерывным параметром, то есть предложенная модель является дискретно-непрерывной.

\subsection{Предложенный метод}

Поиск оптимального решения рассматриваемой функции энергии имеет временную сложность, не допускающую использование его на практике. Также указанная функция энергии содержит и непрерывные и дискретные параметры. Поэтому я предложил итеративный алгоритм поиска локального оптимума. Он сводится к последовательному решению двух задач: 1) определении множества наилучших гипотез позы человека на изображении и 2) оценки скорости движения людей. Предложенный алгоритм основан на последовательной оптимизации групп неизвестных параметров при фиксации остальных.

Допустим, что поза и размер человека зафиксированы на всех кадрах кроме кадра $I_i$.

Предложенный алгоритм базируется на том, что при известных значениях части параметров
\begin{itemize}
	\item если известно оптимальное значение позы и размера человека на всех кадрах, то существует эффективный алгоритм поиска оптимального значения остальных параметров;
	\item если известно оптимальное значение позы и размера человека на всех кадрах, кроме одного, а также известно оптимальное значение скорости его суставов, то существует алгоритм поиска оптимального значения параметров на оставшемся кадре.
\end{itemize}

\begin{algorithm}[H]
	\SetAlgoLined %% Это соединяет линиями логические части 
	%% алгоритма типа if-then-else
	
	\KwData{ $I, K$} %% здесь можно указать исходные параметры
	
	\KwResult{ $P$ } %% результат работы программы
	
	$r \leftarrow \max(I.width(), I.height())$;
	
	$\overline{P} \leftarrow N\_best(I|r)$;
	
	$\overline{V} \leftarrow \argmin{E(V|\overline{P})}$;
	
	\While{ $r > 1$ }{
		
		$t \leftarrow randint(N)$;
		
		$\left\{h_k^p\right\} \leftarrow N\_best\left(I_t | \overline{P}_{\setminus t}, \overline{V}, \right)$;
		
		\For{$k=\overline{1,K}$}{
			
			$H_k^p \leftarrow \left\{\overline{P}_{\setminus t}, h_k^p\right\}$;
			
			$H_k^v \leftarrow \argmin {E(V|H_k^p)}$;
		}
			
		$\left(\widehat{P}, \widehat{V}\right) = \argmin{\left\{E(H_k^p, H_k^v)\right\}}$;
		
		\If{$E(\widehat{P}, \widehat{V}) = E(\overline{P}, \overline{H})$}{
			
			$r \leftarrow \frac{r}{2}$;
			
		}
	
		$\overline{P} \leftarrow \widehat{P}$;

		$\overline{V} \leftarrow \widehat{V}$;
				
	}

	return $\overline{P}$;
	
	\caption{Итеративный алгоритм построения позы человека в видео.}
	\label{alg:generalInit}
\end{algorithm}

\begin{algorithm}[H]
	\SetAlgoLined %% Это соединяет линиями логические части 
	%% алгоритма типа if-then-else
	
	\KwData{ $I, \tilde{P}, N$} %% здесь можно указать исходные параметры
	
	\KwResult{ $P$ } %% результат работы программы
	
	$\tilde{V} \leftarrow \argmax{E(V|\tilde{P})}$;
	
	$H_0 \leftarrow \left(\tilde{P}, \tilde{V} \right)$;
	
	\For{$i=\overline{1,N}$}{
		
		$i \leftarrow sample\_step()$;
		
		$\widehat{P} \leftarrow apply\_step(H_{i-1}, k)$;
		
		$\widehat{V} \leftarrow \argmin{E(V|\widehat{P})}$;
		
		$H_i \leftarrow choose\_next(H_{i-1}, (\widehat{P}, \widehat{V}))$;
		
	}

	$P \leftarrow \argmin_i{E(H_i)}$;
	
	\caption{Алгоритм сэмплирования для построения позы человека в видео.}
	\label{alg:generalHM}
\end{algorithm}

\subsection{Доказательство сходимости}
\subsection{Доказательство корректности}
\subsection{Оценка сложности}
\section{Экспериментальная оценка}
\subsection{Описание тестовых данных}
\subsection{Анализ результатов}
