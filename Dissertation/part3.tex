%\chapter{Вёрстка таблиц} \label{chapt3}

\chapter{Локализация людей на изображении} \label{chapt3}

В данной главе рассматривается задача локализации объектов на изображении. Поза и калибровка камеры накладывают существенные ограничения на возможные размеры и положения объектов на изображении. Например, в системах видеонаблюдения на проезжей части изображения автомобилей не могут полностью находиться выше уровня горизонта. В этой главе представлен метод расширения алгоритмов локализации объектов на изображении информацией о позе и калибровке камеры для увеличения их точности и производительности. Предложенный метод позволяет учесть эти ограничения, и может быть применен к для расширения детектора любого объекта в сцене.

Формально, предложенный алгоритм определяется следующим образом:
\begin{itemize}
	\item[Вход:]
	\begin{enumerate}
		\item Изображение $I$;
		\item Параметры позы $l_c$ камеры (см. главу \ref{chapt2});
		\item Фокусное расстояние $f$ камеры;
		\item Базовый детектор объектов на изображении $d_b$;
	\end{enumerate}
	\item[Выход:] Множество прямоугольников, ограничивающих изображения объектов интереса.
\end{itemize}

Ключевой идеей предложенного метода является классификация результатов локализации объектов на ``правдоподобные'' и ``невозможные'' с геометрической точки зрения. Он состоит из двух этапов: 1) локализация объектов базовым алгоритмом $d_b$; 2) классификация результатов локализации. Предложенный метод может быть интегрирован с любым методом локализации объектов на изображении. При интеграции с алгоритмами, использующие скользящее окно для локализации, предложенный метод позволяет также увеличить производительность базового алгоритма. В этой работе я показал способ его применения для увеличения производительности и точности базового алгоритма локализации голов людей на изображении \cite{prisacariu2009fasthog}.

Для применения предложенного метода локализации объектов интереса необходимо решить 3 задачи:
\begin{enumerate}
	\item Построить синтетическую выборку наблюдаемых данных;
	\item Построить признаки, инвариантные для синтетических и реальных данных;
	\item Обучение классификатора результатов работы алгоритма локализации объектов.
\end{enumerate}
В дальнейших разделах будет подробно описано решение всех этих задач.

\section{Математическая модель наблюдаемых данных}

Я использую модель наблюдаемых данных, описанную в \ref{chap-cam_pose::sec-model}.

\section{Предложенный метод}

Я предложил метод расширения методов локализации объектов на изображении информацией о позе и параметрах камеры в сцене. Предложенный метод позволяет учитывать физические ограничения на положение объектов в сцене при их локализации на изображении. Для решения этой задачи я предложил метод классификации обнаружений базового алгоритма $d_b$ на ``правдоподобные'' и ``невозможные''.

\subsection{Построение обучающей выборки}

Обучение классификатора на реальных данных оказывается затруднительным из-за малого количества размеченных данных. Поэтому была использована синтетическая выборка, описанная в разделе \ref{chapt2::sect_dataset}. В качестве признакового описания результатов обнаружения использовались параметры 1) обнаруженного региона и 2) используемой камеры. Таким образом, входом алгоритма классификации является вектор, состоящий из:
\begin{itemize}
	\item результат обнаружения $o = (x_o, y_o, s_o)$ объекта базовым алгоритмом $d_b$;
	\item поза $l_c = (h, t, r)$ камеры;
	\item фокусное расстояние $f$ камеры.
\end{itemize}

Формально задача классификации срабатываний детектора по построенной выборке относится к классу задач поиска аномалий. Отличительной особенностью таких задач является отсутствие примеров отрицательного класса (ошибки детектора) в обучающей выборке. В своей работе я предложил способ моделирования объектов такого класса. Для их построения я объединил две стратегии: 1) использование обнаружений, характерных для других последовательностей обучающей выборки; 2) использование регионов изображения с произвольным положением и размером. Первая стратегия позволяет обучать классификатор, дискриминативный к позе и калибровке камеры. Вторая стратегия позволяет классификатору отличать верные обнаружения от ложных срабатываний базового детектора $d_p$. При построении отрицательных примеров обучающей выборки эти стратегии использовались в отношении 1:9.

\subsection{Построение классификатора}

Для построения классификатора обнаружений детектора мы использовали нейронную сеть, представленную на рисунке \ref{}. Сеть состоит из последовательности полносвязных слоёв и нелинейной функциии ReLu.

\section{Обучение и экспериментальная оценка}

\subsection{Обучение}

Я обучил предложенную сеть на синтетической выборке, состоящей из 24143 сцен, с помощью библиотеки caffe \cite{CITATION jia2014caffe \l 1033}. Скорость обучения понижалась каждые 1500 итераций со степенной скоростью $\gamma = 0.95$.

\subsection{Экспериментальная оценка на синтетической выборке}

Я оценили качество предложенного алгоритма на синтетических и реальных данных, а также интегрировали его с используемым детектором.

Мы провели тестирование на синтетической выборке. Наша тестовая выборка состоит из 6036 различных сцен, не использованных при обучении. На рисунке \ref{}~(а) представлено качество классификации обнаружений на синтетической тестовой выборке. 

\subsection{Экспериментальная оценка на реальных данных}

Для проведения тестирования необходимо знать положение камеры для каждого кадра выборки. Это затрудняет использование стандартных баз размеченных изображений.

Я использовал выборку TownCentre \cite{benfold2011stable}. Были использованы параметры позы камеры, предсказанные методом из главы \ref{chapt2}. При классификации результатов локализации объектов базовым алгоритмом $d_b$ я использовал параметр 0.25. Обнаружение считалось правильным, если его пересечение с регионом головы человека в разметке занимало не менее 25\% их объединения. Сравнение качества исходного и полученного детектора (см. рис.~\ref{}(б)) выявило, что предложенный алгоритм фильтрации позволяет увеличить точность без существенного уменьшения полноты базового алгоритма локализации.

\subsection{Интеграция с алгоритмом детектирования}

Я показал, как предложенный метод классификации можно использовать для увеличения производительности базового детектора $d_b$. Построенный классификатор задает статические ограничения на допустимые положения объектов на изображении. Эта идея продемонстрирована для позы камеры соответствующей выборке TownCentre на рисунке~\ref{} (первая строка). Таким образом, для каждого размера объекта на изображении, можно указать, регион, где он может быть обнаружен.

Эта информация может быть использована для увеличения производительности базового алгоритма локализации. Если базовый алгоритм использует пирамиду изображений, то для каждого её уровня обрабатывать необходимо лишь небольшой регион, предсказанный классификатором. Для алгоритмов, производящих регрессию заранее заданных прямоугольников, для каждого региона изображения классификатор предсказывает, какие прямоугольники не могут соответствовать правдоподобным положениям объектов интереса.

В этой работе я применил этот метод для увеличения производительности базового детектора \cite{prisacariu2009fasthog}. При тестировании на выборке TownCentre классификатор указывает, что необходимо обработать лишь 21.44\% регионов на всех уровнях пирамиды. В предложенной реализации алгоритм обрабатывает прямоугольные регионы изображения на каждом уровне пирамиды (см. рис.~\ref{} вторая строка). Это привело к обработке 24.03\% всех регионов, но упростило реализацию. Такой подход позволил увеличит производительность алгоритма локализации изображений голов людей с 20.03 до 34.36 кадров в секунду на выборке TownCentre.