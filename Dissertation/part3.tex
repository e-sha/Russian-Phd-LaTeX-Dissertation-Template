%\chapter{Вёрстка таблиц} \label{chapt3}

\chapter{Локализация людей на изображении} \label{chapt3}

В данной главе рассматривается задача локализации объектов на изображении. Классические алгоритмы работают в  Мною предлагается метод развития методов  обнаружений детектора голов человека на корректные и «невозможные» с геометрической точки зрения. Построенный алгоритм опирается на информацию о положении и параметрах камеры. Важно отметить, что ключевым требованием к разработанному классификатору является инвариантность к классификатору окон, используемому детектором. Это позволяет построить алгоритм фильтрации обнаружений для любого детектора. Формально, входом нашего алгоритма являет положение камеры и признаки классифицируемого обнаружения. Под обнаружением мы понимаем прямоугольник, ограничивающий изображение объекта, найденного детектором.

Предложенный метод построения классификатора результатов детектора состоит трех этапов:

1. Построение синтетической выборки изображений;

2. Построение признаков обнаружений, инвариантных для синтетических и реальных данных;

3. Обучение классификатора.

Ниже мы рассмотрим подробнее каждый их этих этапов для построения классификатора обнаружений детектора голов человека.

\section{Математическая модель наблюдаемых данных}

Построение обучающей выборки

Для обучения классификатора обнаружений необходима обучающая выборка. Мы не могли использовать размеченные данные видеонаблюдения из-за сложности их сбора. Поэтому, используя компьютерную графику, мы построили синтетическую выборку, моделирующую сценарий видеонаблюдения. Выборка состоит из 30179 сцен, каждая из которых определяется положением камеры. Каждая сцена состоит из плоскости земли, стоящих на ней людей и камеры. Мы использовали упрощенную модель камеры, которая определяется углами наклона и поворота камеры, фокусным расстоянием и высотой камеры над плоскостью земли. Для построения каждой сцены параметры камеры выбирались случайным образом из равномерного распределения (см. Таблица 1).

Таблица 1 Параметры камеры (Camera parameters).
Обозначение
Параметр
Диапазон значений
h
Высота

P
Наклон

R
Поворот

F
Фокусное расстояние

Мы использовали результаты работы [ CITATION pishchulin2015building \l 1033 ] для моделирования людей в сцене. Каждая построенная сцена содержит не менее 200 человек, размещенных в случайных положениях плоскости земли.

\section{Предложенный метод}

Обучение алгоритмов машинного обучения только на синтетической выборке может привести к эффекту переобучения. Поэтому важным этапом обработки является построение признаков инвариантных к типу обрабатываемых данных. В качестве таких признаков мы выбрали параметры ограничивающего прямоугольника обнаруженного детектором. Для задачи фильтрации обнаружений головы мы использовали оптимизированную версию детектора голов людей [ CITATION prisacariu2009fasthog \l 1033 ] из-за высокой скорости его работы.

Модель [ CITATION pishchulin2015building \l 1033 ] позволила нам исключить ложные обнаружения из обучающей выборке за счет информации о положении ключевых точек на человеке.

Построение классификатора

Формально задача классификации обнаружений ложных срабатываний детектора по построенной выборке относится к классу задач поиска аномалий. Отличительной особенностью таких задач является отсутсвие примеров отрицательного класса (ошибки детектора). Мы сводим нашу задачу к задаче классификации. Мы предложили простую модель распределения ошибочных результатов работы детектора, состоящую из смеси двух распределений.

Первое моделирует случайные ошибки детектора. Для этого используется равномерное распределение обнаружений (их положений и размеров) по изображению. В качестве второго компонента смеси мы взяли распределение обнаружений в построенной выборке. Это позволяет алгоритму отличить обнаружения, характерные сцен. Наши эксперименты показали, что наилучшие результаты классификации получаются при смешивание этих распределений в отношении 1:9.

Для построения классификатора обнаружений детектора мы использовали нейронную сеть, представленную на Рисунок 3. Сеть состоит из полносвязных слоев, после которых используется нелинейная функция ReLu.

\section{Обучение и экспериментальная оценка}

\subsection{Обучение}

Я обучили предложенную сеть на синтетической выборке, состоящей из 24143 сцен, с помощью библиотеки caffe [ CITATION jia2014caffe \l 1033 ]. Скорость обучения понижалась каждые 1500 итераций с параметром , равным 0.95.

Я оценили качество предложенного алгоритма на синтетических и реальных данных, а также интегрировали его с используемым детектором.

\subsection{Экспериментальная оценка на синтетической выборке}

Мы провели тестирование на синтетической выборке. Наша тестовая выборка состоит из 6036 различных сцен, не использованных при обучении. На Рисунок 2 (а) представлена качество классификации обнаружений на синтетической тестовой выборке. 

\subsection{Экспериментальная оценка на реальных данных}

Для проведения тестирования необходимо знать положение камеры для каждого кадра выборки. Это затрудняет использование стандартных баз размеченных изображений.

Мы использовали выборку TownCentre [ CITATION benfold2011stable \l 1033 ], так как она содержит большую часть необходимой информации. К сожалению, параметры положения камеры, представленные в выборке, оказали неправдоподобными. Это может быть связано с использованием единицы измерения в мировой системе координат, отличной от метра. Мы заметили, что высота камеры над землей приблизительно равна 8 метрам в данной сцене.

Мы применили фильтрацию обнаружений детектора с порогом 0.25. Обнаружение считалось правильным, если его пересечение с регионом головы человека в разметке занимало не менее 25\% их объединения. Сравнение качества исходного и полученного детектора (Рисунок 2 (б)) выявило, что предложенный алгоритм фильтрации позволяет увеличить точность без существенного уменьшения полноты обнаружения.

\subsection{Интеграция с алгоритмом детектирования}

Мы использовали построенный фильтр обнаружений детектора для ускорения его работы. Действительно, можно оценить, какую область изображения необходимо обрабатывать на каждом кадре.

Мы построили маски областей, которые необходимо обрабатывать на каждом уровне пирамиды изображений для камеры, соответствующей выборке TownCentre ( Рисунок 4 первая строка). Результаты показывают, в сцене необходимо обработать только небольшое подмножество окон. Например, для сцены TownCentre достаточно обработать лишь 21.44\% всех окон.

Мы расширили базовый детектор возможностью обрабатывать только те строки изображения на каждом уровне пирамиды, где правдоподобно обнаружение головы человека. Пример обрабатываемых областей представлен на  Рисунок 4 (вторая строка). Это соответствует обработке 24.03\% всех окон. 

Выделение областей интереса является эффективным способом повышения производительности детектора. Обработка области интереса на сцене TownCentre позволило повысить скорость обработки данных с 20.03 до 34.36 кадров в секунду.

Полученный результат наиболее важен для систем видеонаблюдения, где параметры камеры резко изменяются и могут быть оценены один раз.

\iffalse
\chapter{Определение позы человека в видео}
\fi

\section{Таблица обыкновенная} \label{sect3_1}

Так размещается таблица:

\begin{table} [htbp]
  \centering
  \captionsetup{width=15cm}
  \caption{Название таблицы}\label{Ts0Sib}%
  \begin{tabular}{| p{3cm} || p{3cm} | p{3cm} | p{4cm}l |}
  \hline
  \hline
  Месяц   & \centering $T_{min}$, К & \centering $T_{max}$, К &\centering  $(T_{max} - T_{min})$, К & \\
  \hline
  Декабрь &\centering  253.575   &\centering  257.778    &\centering      4.203  &   \\
  Январь  &\centering  262.431   &\centering  263.214    &\centering      0.783  &   \\
  Февраль &\centering  261.184   &\centering  260.381    &\centering     $-$0.803  &   \\
  \hline
  \hline
  \end{tabular}
\end{table}

\begin{table} [htbp]% Пример записи таблицы с номером, но без отображаемого наименования
	\centering
	\parbox{9cm}{% чтобы лучше смотрелось, подбирается самостоятельно
        \captionsetup{format=tablenocaption}% должен стоять до самого caption
        \caption{}%
        \label{tbl:test1}%
        \begin{SingleSpace}
    	\begin{tabular}{ | c | c | c | c |}
    	\hline
    	Оконная функция	& ${2N}$ & ${4N}$	& ${8N}$	\\ \hline
    	Прямоугольное 	& 8.72 	 & 8.77		& 8.77		\\ \hline
    	Ханна		& 7.96 	 & 7.93		& 7.93		\\ \hline
    	Хэмминга	& 8.72 	 & 8.77		& 8.77		\\ \hline
    	Блэкмана	& 8.72 	 & 8.77		& 8.77		\\ \hline
    	\end{tabular}%
    	\end{SingleSpace}
	}
\end{table}

Таблица \ref{tbl:test2} "--- пример таблицы, оформленной в~классическом книжном варианте или~очень близко к~нему. \mbox{ГОСТу} по~сути не~противоречит. Можно ещё~улучшить представление, с~помощью пакета \verb|siunitx| или~подобного.

\begin{table} [htbp]%
    \centering
	\caption{Наименование таблицы, очень длинное наименование таблицы, чтобы посмотреть как оно будет располагаться на~нескольких строках и~переноситься}%
	\label{tbl:test2}% label всегда желательно идти после caption
    \renewcommand{\arraystretch}{1.5}%% Увеличение расстояния между рядами, для улучшения восприятия.
    \begin{SingleSpace}
	\begin{tabular}{@{}@{\extracolsep{20pt}}llll@{}} %Вертикальные полосы не используются принципиально, как и лишние горизонтальные (допускается по ГОСТ 2.105 пункт 4.4.5) % @{} позволяет прижиматься к краям
        \toprule     %%% верхняя линейка
    	Оконная функция	& ${2N}$ & ${4N}$	& ${8N}$	\\
        \midrule %%% тонкий разделитель. Отделяет названия столбцов. Обязателен по ГОСТ 2.105 пункт 4.4.5 
    	Прямоугольное 	& 8.72 	 & 8.77		& 8.77		\\
    	Ханна		& 7.96 	 & 7.93		& 7.93		\\
    	Хэмминга	& 8.72 	 & 8.77		& 8.77		\\
    	Блэкмана	& 8.72 	 & 8.77		& 8.77		\\
        \bottomrule %%% нижняя линейка
	\end{tabular}%
   	\end{SingleSpace}
\end{table}

\section{Таблица с многострочными ячейками и примечанием}

Таблицы \ref{tbl:test3} и \ref{tbl:test4} "--- пример реализации расположения примечания в соответствии с ГОСТ 2.105. Каждый вариант со своими достоинствами и недостатками. Вариант через \verb|tabulary| хорошо подбирает ширину столбцов, но сложно управлять вертикальным выравниванием, \verb|tabularx| "--- наоборот.
\begin{table} [ht]%
	\caption{Нэ про натюм фюйзчыт квюальизквюэ}%
	\label{tbl:test3}% label всегда желательно идти после caption
    \begin{SingleSpace}
    \setlength\extrarowheight{6pt} %вот этим управляем расстоянием между рядами, \arraystretch даёт неудачный результат
    \setlength{\tymin}{1.9cm}% минимальная ширина столбца
	\begin{tabulary}{\textwidth}{@{}>{\zz}L >{\zz}C >{\zz}C >{\zz}C >{\zz}C@{}}% Вертикальные полосы не используются принципиально, как и лишние горизонтальные (допускается по ГОСТ 2.105 пункт 4.4.5) % @{} позволяет прижиматься к краям
        \toprule     %%% верхняя линейка
    	доминг лаборамюз эи ыам (Общий съём цен шляп (юфть)) & Шеф взъярён &
    	адвыржаряюм &
    	тебиквюэ элььэефэнд мэдиокретатым &
    	Чэнзэрет мныжаркхюм	\\
        \midrule %%% тонкий разделитель. Отделяет названия столбцов. Обязателен по ГОСТ 2.105 пункт 4.4.5 
         Эй, жлоб! Где туз? Прячь юных съёмщиц в~шкаф Плюш изъят. Бьём чуждый цен хвощ! &
        ${\approx}$ &
        ${\approx}$ &
        ${\approx}$ &
        $ + $ \\
        Эх, чужак! Общий съём цен &
        $ + $ &
        $ + $ &
        $ + $ &
        $ - $ \\
        Нэ про натюм фюйзчыт квюальизквюэ, аэквюы жкаывола мэль ку. Ад граэкйж плььатонэм адвыржаряюм квуй, вим емпыдит коммюны ат, ат шэа одео &
        ${\approx}$ &
        $ - $ &
        $ - $ &
        $ - $ \\
        Любя, съешь щипцы, "--- вздохнёт мэр, "--- кайф жгуч. &
        $ - $ &
        $ + $ &
        $ + $ &
        ${\approx}$ \\
        Нэ про натюм фюйзчыт квюальизквюэ, аэквюы жкаывола мэль ку. Ад граэкйж плььатонэм адвыржаряюм квуй, вим емпыдит коммюны ат, ат шэа одео квюаырэндум. Вёртюты ажжынтиор эффикеэнди эож нэ. &
        $ + $ &
        $ - $ &
        ${\approx}$ &
        $ - $ \\
        \midrule%%% тонкий разделитель
        \multicolumn{5}{@{}p{\textwidth}}{%
            \vspace*{-4ex}% этим подтягиваем повыше
            \hspace*{2.5em}% абзацный отступ - требование ГОСТ 2.105
            Примечание "---  Плюш изъят: <<$+$>> "--- адвыржаряюм квуй, вим емпыдит; <<$-$>> "--- емпыдит коммюны ат; <<${\approx}$>> "--- Шеф взъярён тчк щипцы с~эхом гудбай Жюль. Эй, жлоб! Где туз? Прячь юных съёмщиц в~шкаф. Экс-граф?
        }
        \\
        \bottomrule %%% нижняя линейка
	\end{tabulary}%
    \end{SingleSpace}
\end{table}

Из-за того, что таблица \ref{tbl:test3} не помещается на той же странице (при компилировании pdflatex), всё её содержимое переносится на следующую, ближайшую, а этот текст идёт перед ней.
\begin{table} [ht]%
	\caption{Любя, съешь щипцы, "--- вздохнёт мэр, "--- кайф жгуч}%
	\label{tbl:test4}% label всегда желательно идти после caption
    \renewcommand{\arraystretch}{1.6}%% Увеличение расстояния между рядами, для улучшения восприятия.
	\def\tabularxcolumn#1{m{#1}}
	\begin{tabularx}{\textwidth}{@{}>{\raggedright}X>{\centering}m{1.9cm} >{\centering}m{1.9cm} >{\centering}m{1.9cm} >{\centering\arraybackslash}m{1.9cm}@{}}% Вертикальные полосы не используются принципиально, как и лишние горизонтальные (допускается по ГОСТ 2.105 пункт 4.4.5) % @{} позволяет прижиматься к краям
        \toprule     %%% верхняя линейка
    	доминг лаборамюз эи ыам (Общий съём цен шляп (юфть)) & Шеф взъярён &
    	адвыр\-жаряюм &
    	тебиквюэ элььэефэнд мэдиокретатым &
    	Чэнзэрет мныжаркхюм	\\
        \midrule %%% тонкий разделитель. Отделяет названия столбцов. Обязателен по ГОСТ 2.105 пункт 4.4.5 
         Эй, жлоб! Где туз? Прячь юных съёмщиц в~шкаф Плюш изъят. Бьём чуждый цен хвощ! &
        ${\approx}$ &
        ${\approx}$ &
        ${\approx}$ &
        $ + $ \\
        Эх, чужак! Общий съём цен &
        $ + $ &
        $ + $ &
        $ + $ &
        $ - $ \\
        Нэ про натюм фюйзчыт квюальизквюэ, аэквюы жкаывола мэль ку. Ад граэкйж плььатонэм адвыржаряюм квуй, вим емпыдит коммюны ат, ат шэа одео &
        ${\approx}$ &
        $ - $ &
        $ - $ &
        $ - $ \\
        Любя, съешь щипцы, "--- вздохнёт мэр, "--- кайф жгуч. &
        $ - $ &
        $ + $ &
        $ + $ &
        ${\approx}$ \\
        Нэ про натюм фюйзчыт квюальизквюэ, аэквюы жкаывола мэль ку. Ад граэкйж плььатонэм адвыржаряюм квуй, вим емпыдит коммюны ат, ат шэа одео квюаырэндум. Вёртюты ажжынтиор эффикеэнди эож нэ. &
        $ + $ &
        $ - $ &
        ${\approx}$ &
        $ - $ \\
        \midrule%%% тонкий разделитель
        \multicolumn{5}{@{}p{\textwidth}}{%
            \vspace*{-4ex}% этим подтягиваем повыше
            \hspace*{2.5em}% абзацный отступ - требование ГОСТ 2.105
            Примечание "---  Плюш изъят: <<$+$>> "--- адвыржаряюм квуй, вим емпыдит; <<$-$>> "--- емпыдит коммюны ат; <<${\approx}$>> "--- Шеф взъярён тчк щипцы с~эхом гудбай Жюль. Эй, жлоб! Где туз? Прячь юных съёмщиц в~шкаф. Экс-граф?
        }
        \\
        \bottomrule %%% нижняя линейка
	\end{tabularx}%
\end{table}

%\newpage
%============================================================================================================================

\section{Параграф - два} \label{sect3_2}

Некоторый текст.

%\newpage
%============================================================================================================================

\section{Параграф с подпараграфами} \label{sect3_3}

\subsection{Подпараграф - один} \label{subsect3_3_1}

Некоторый текст.

\subsection{Подпараграф - два} \label{subsect3_3_2}

Некоторый текст.

\clearpage