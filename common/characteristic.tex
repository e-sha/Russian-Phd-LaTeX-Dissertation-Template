% !TeX spellcheck = ru_RU

{\actuality}

В современном мире системы видеонаблюдения становятся важной частью инфраструктуры городов и предприятий. Дорожные камеры автоматически фиксируют наружения правил дорожного движения, принуждая водителей им следовать. Системы видеонаблюдения на улицах городов и общественном транспорте сохраняют записи происходящего, позволяя восстановить действительную картину случившегося при анализе конфликтных ситуаций и правонарушений. Системы идентификации человека по лицу активно внедряются для упрощения процедуры прохождения паспортного контроля в аэропортах. Коммерческие организации внедряют видеонаблюдение также для улучшения качества обслуживания путем анализа поведения клиентов (например, маршрутов движения покупателей в крупных магазинах).

Несмотря на эти примеры использования, потенциальные возможности видеонаблюдения существенно шире. В отличие от человека современные алгоритмы анализа видео не способны провести анализ поведения людей в наблюдаемой сцене. Это ограничивает возможность предотвращения правонарушений на ранней стадии. Также современные алгоритмы существенно уступают человеку в качестве сопровождения людей в видео. Несмотря на то, что для человека эта задача не составляет труда, даже в случае слабо загруженных сцен существующие методы допускают ошибки при сопровождении. Эта проблема ограничивает возможность использования видеонаблюдения для поиска людей. Например, в настоящий момент только ручной анализ видеоданных оператором может позволить достоверно определить местоположение пропавшего человека.

Другим существенным ограничением развития видеонаблюдения является высокая вычислительная сложность современных алгоритмов анализа видео. Широкая доступность видеокамер и развитие компьютерных сетей позволили создать системы видеонаблюдения, объединяющие более сотни тысяч камер. Однако даже алгоритмы первичного анализа, такие как обнаружение объектов интереса, не позволяют обрабатывать больше нескольких видеопотоков на центральном процессоре или рассчитаны на дорогостоящие графические ускорители.

Таким образом, для развития систем видеонаблюдения необходимо разработать алгоритмы анализа, превосходящие существующие по точности и качеству. В своей работе я ограничиваюсь рассмотрением задачи анализа данных систем видеонаблюдения, включающих единственную неподвижную камеру. В рамках такой постановки стандартный подход \cite{wang2013intelligent} к анализу данных видеонаблюдения заключается в решении следующих подзадач:
\begin{enumerate}
	\item Калибровка камеры (построение отображения между мировой системой координат и системой координат изображения);
	\item обнаружение и сопровождение объектов интереса (например, людей) в видео;
	\item анализ поведения (подразумевает автоматическое определение типа поведения и выявление аномального поведения).
\end{enumerate}

Несмотря на активные исследования описанных задач, широкое практическое применение получили только методы обнаружения объектов интереса (начиная с работы \cite{viola2001rapid}) и идентификации человека по изображению лица (например, \cite{taigman2014deepface,schroff2015facenet}). Последние однако опираются на предположение, что лицо занимает значительную часть изображения "--- $152\times152$ пикселей. Это предположение не позволяет использовать предложенные методы во многих практических системах видеонаблюдения, где размер изображения тела человека по любой из координат может не превосходить 200 пикселей. Таким образом, из-за описанных проблем многие современные системы видеонаблюдения остаются системами сбора и хранения видеоданных.

Поэтому в работе предложены алгоритмы и методики работы с данными видеонаблюдения, содержащие изображения людей даже небольшого размера. К таким относятся данные, где размер изображения головы человека имеет размер от $24\times24$ пикселя.

% {\progress} 
% Этот раздел должен быть отдельным структурным элементом по
% ГОСТ, но он, как правило, включается в описание актуальности
% темы. Нужен он отдельным структурынм элемементом или нет ---
% смотрите другие диссертации вашего совета, скорее всего не нужен.

{\aim} данной работы является разработка методов повышения качества локализации, сопровождения и определения позы людей в видеопоследовательности, полученных статичной камерой, за счёт использования информации о калибровке камеры и движении людей в сцене.

Для~достижения поставленной цели необходимо было решить следующие {\tasks}:
\begin{enumerate}
  \item Разработать алгоритм локализации камеры, основанного на длительном наблюдении за объектами в сцене;
  \item Разработать алгоритм расширения методов локализации людей на изображении информацией о семантики сцены для повышения их точности;
  \item Разработать обобщение модели позы человека в видеопоследовательности, допускающее задание модели движения человека.
\end{enumerate}

{\defpositions}
\begin{enumerate}
  \item Предложен и реализован новый алгоритм уточнения результатов обнаружения объектов интереса на изображении, использующий информацию о калибровке камеры, основанный на обучении по синтетическим прецедентам. Представлен способ интеграции предложенного алгоритма в существующий метод обнаружения голов людей, использующие подход скользящего окна. Эмпирически показана применимость предложенного метода для увеличения производительности и качества обнаружения для широкого диапазона значений параметров калибровки камеры.
  \item Предложен и реализован новый алгоритм сопровождения людей в видеопоследовательности, использующий устойчивый метод построения треклетов и информацию о регионах входа/выхода для заданной сцены. Превосходство предложенного метода перед аналогами показано путем сравнения результатов с эталоном.
  \item Предложено обобщение модели позы человека в видео моделью движения суставов в виде линейной динамической системы. Предложен и реализован алгоритм поиска локального оптимума представленной модели. Превосходство предложенной модели перед аналогами показано путем сравнения результатов с эталоном. 
\end{enumerate}

{\novelty}
\begin{enumerate}
  \item Впервые предложен способ построения отображения размеров и положений объектов на изображении в позу камеры, основанный на обучении по синтетическим прецедентам, позволяющий определять параметры позы камеры в широком диапазоне. Ранее существовавшие работы предлагали аналитические модели отображения и явно использовали предположение о близости угла наклона камеры к горизонтальному и/или отсутствию поворота камеры относительно оптической оси (крена камеры).
  \item Впервые предложен алгоритм уточнения результатов обнаружения объектов на изображении, использующий информацию о калибровке камеры, основанный на обучении по синтетическим прецедентам, и допускающий широкий диапазон значений параметров калибровки. Ранее существовавшие алгоритмы делали предположение о наклоне камеры близкому к горизонтальному.
  \item Впервые было предложены расширение модели позы человека в видео информацией о его движении в виде линейной динамической системы и эффективный алгоритм поиска локального оптимума предложенной модели. Ранее существовавшие модели рассматривали изменение позы человека между кадрами как допустимое отклонение от позы на соседних кадрах, то есть не учитывали скорость движения человека.
\end{enumerate}

{\influence} Предложенный алгоритм уточнения результатов обнаружения объектов интереса был применен к алгоритму обнаружения голов людей на изображении и интегрирован в него. Эмпирическая оценка показала, что полученный алгоритм превосходит базовый одновременно и в скорости обработки данных и в точности обнаружения. В работе представлен способ интеграции с произвольным алгоритмом обнаружения объектов, использующем метод скользящего окна. Полученный алгоритм обнаружения требует задания параметров калибровки в виде фокусного расстояния и позы камеры в сцене. Однако, в работе также представлен метод определения позы камеры, использующий результаты обнаружения объектов в сцене. Интеграция обоих предложенных алгоритмов в алгоритм обнаружения объектов интереса позволяет уменьшить количество требуемых параметров только до фокусного расстояния оптической системы камеры. Дальнейшее объединение с существующими методами оценки фокусного расстояния позволит создать полностью автоматический алгоритм обнаружения объектов интереса, использующий информацию о калибровке камеры для обработки только релевантных регионов изображения.



%{\methods} \ldots

{\reliability} полученных результатов обеспечивается \ldots \ Результаты находятся в соответствии с результатами, полученными другими авторами.


{\probation}
Основные результаты работы докладывались~на:
\begin{itemize}
	\item CMC MSU-Huawei International Workshop "Selected topics in multimedia image processing and analysis" (Москва, Россия, 31 августа 2016);
	\item 5th International Workshop on Image Mining. Theory and Applications (Берлин, Германия, 2015 год);
	\item 11th International Conference on Pattern Recognition and Image Analysis: New Information Technologies (Самара, Россия, 2013 год);
	\item 26-я Международная конференция по компьютерной графике, обработке изображений и машинному зрению, системам визуализации и виртуального окружения GraphiCon 2016 (Нижний Новгород, Россия, 19-23 сентября 2016 год);
	\item 25-я Международная конференция по компьютерной графике, обработке изображений и машинному зрению, системам визуализации и виртуального окружения GraphiCon 2015 (Протвино, Россия, 22-25 сентября 2015 год);	
	\item 24-я Международная конференция по компьютерной графике, обработке изображений и машинному зрению, системам визуализации и виртуального окружения GraphiCon 2014 (Ростов-на-Дону, Россия, 30 сентября-3 октября 2014 год);
\end{itemize}

{\contribution} Автор принимал активное участие \ldots

%\publications\ Основные результаты по теме диссертации изложены в ХХ печатных изданиях~\cite{Sokolov,Gaidaenko,Lermontov,Management},
%Х из которых изданы в журналах, рекомендованных ВАК~\cite{Sokolov,Gaidaenko}, 
%ХХ --- в тезисах докладов~\cite{Lermontov,Management}.

\ifnumequal{\value{bibliosel}}{0}{% Встроенная реализация с загрузкой файла через движок bibtex8
    \publications\ Основные результаты по теме диссертации изложены в XX печатных изданиях, 
    X из которых изданы в журналах, рекомендованных ВАК, 
    X "--- в тезисах докладов.%
}{% Реализация пакетом biblatex через движок biber
%Сделана отдельная секция, чтобы не отображались в списке цитированных материалов
    \begin{refsection}%
        \printbibliography[heading=countauthornotvak, env=countauthornotvak, keyword=biblioauthornotvak, section=1]%
        \printbibliography[heading=countauthorvak, env=countauthorvak, keyword=biblioauthorvak, section=1]%
        \printbibliography[heading=countauthorconf, env=countauthorconf, keyword=biblioauthorconf, section=1]%
        \printbibliography[heading=countauthor, env=countauthor, keyword=biblioauthor, section=1]%
        \publications\ Основные результаты по теме диссертации изложены в \arabic{citeauthor} печатных изданиях, 
        \arabic{citeauthorvak} из которых изданы в журналах, рекомендованных ВАК, 
        \nocite{konushin2017ispolzovanie48435059}
        \nocite{konushin2015an-improvement33378280}
        \nocite{konushin2015human9613323}
        \nocite{gringauz2016estimation34131922}
        \nocite{konushin2013improvement5112417}
        
        \arabic{citeauthorconf} "--- в тезисах докладов.
        
        \nocite{gringauz2015an-algorithm10904785}
        \nocite{konushin2016further34136983}
        \nocite{gringauz2014modification7624365}
    \end{refsection}
}
При использовании пакета \verb!biblatex! для автоматического подсчёта
количества публикаций автора по теме диссертации, необходимо
их здесь перечислить с использованием команды \verb!\nocite!.
