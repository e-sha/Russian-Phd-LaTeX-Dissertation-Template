% !TeX spellcheck = ru_RU

\section*{Общая характеристика работы}

\newcommand{\actuality}{\underline{\textbf{\actualityTXT}}}
\newcommand{\progress}{\underline{\textbf{\progressTXT}}}
\newcommand{\aim}{\underline{{\textbf\aimTXT}}}
\newcommand{\tasks}{\underline{\textbf{\tasksTXT}}}
\newcommand{\novelty}{\underline{\textbf{\noveltyTXT}}}
\newcommand{\influence}{\underline{\textbf{\influenceTXT}}}
\newcommand{\methods}{\underline{\textbf{\methodsTXT}}}
\newcommand{\defpositions}{\underline{\textbf{\defpositionsTXT}}}
\newcommand{\reliability}{\underline{\textbf{\reliabilityTXT}}}
\newcommand{\probation}{\underline{\textbf{\probationTXT}}}
\newcommand{\contribution}{\underline{\textbf{\contributionTXT}}}
\newcommand{\publications}{\underline{\textbf{\publicationsTXT}}}

% !TeX spellcheck = ru_RU

{\actuality}

В современном мире системы видеонаблюдения становятся важной частью инфраструктуры городов и предприятий. Дорожные камеры автоматически фиксируют наружения правил дорожного движения, принуждая водителей им следовать. Системы видеонаблюдения на улицах городов и общественном транспорте сохраняют записи происходящего, позволяя восстановить действительную картину случившегося при анализе конфликтных ситуаций и правонарушений. Системы идентификации человека по лицу активно внедряются для упрощения процедуры прохождения паспортного контроля в аэропортах. Коммерческие организации внедряют видеонаблюдение также для улучшения качества обслуживания путем анализа поведения клиентов (например, маршрутов движения покупателей в крупных магазинах).

Несмотря на эти примеры использования, потенциальные возможности видеонаблюдения существенно шире. В отличие от человека современные алгоритмы анализа видео не способны провести анализ поведения людей в наблюдаемой сцене. Это ограничивает возможность предотвращения правонарушений на ранней стадии. Также современные алгоритмы существенно уступают человеку в качестве сопровождения людей в видео. Несмотря на то, что для человека эта задача не составляет труда, даже в случае слабо загруженных сцен существующие методы допускают ошибки при сопровождении. Эта проблема ограничивает возможность использования видеонаблюдения для поиска людей. Например, в настоящий момент только ручной анализ видеоданных оператором может позволить достоверно определить местоположение пропавшего человека.

Другим существенным ограничением развития видеонаблюдения является высокая вычислительная сложность современных алгоритмов анализа видео. Широкая доступность видеокамер и развитие компьютерных сетей позволили создать системы видеонаблюдения, объединяющие более сотни тысяч камер. Однако даже алгоритмы первичного анализа, такие как обнаружение объектов интереса, не позволяют обрабатывать больше нескольких видеопотоков на центральном процессоре или рассчитаны на дорогостоящие графические ускорители.

Таким образом, для развития систем видеонаблюдения необходимо разработать алгоритмы анализа, превосходящие существующие по точности и качеству. В своей работе я ограничиваюсь рассмотрением задачи анализа данных систем видеонаблюдения, включающих единственную неподвижную камеру. В рамках такой постановки стандартный подход \cite{wang2013intelligent} к анализу данных видеонаблюдения заключается в решении следующих подзадач:
\begin{enumerate}
	\item Калибровка камеры (построение отображения между мировой системой координат и системой координат изображения);
	\item обнаружение и сопровождение объектов интереса (например, людей) в видео;
	\item анализ поведения (подразумевает автоматическое определение типа поведения и выявление аномального поведения).
\end{enumerate}

Несмотря на активные исследования описанных задач, широкое практическое применение получили только методы обнаружения объектов интереса (начиная с работы \cite{viola2001rapid}) и идентификации человека по изображению лица (например, \cite{taigman2014deepface,schroff2015facenet}). Последние однако опираются на предположение, что лицо занимает значительную часть изображения "--- $152\times152$ пикселей. Это предположение не позволяет использовать предложенные методы во многих практических системах видеонаблюдения, где размер изображения тела человека по любой из координат может не превосходить 200 пикселей. Таким образом, из-за описанных проблем многие современные системы видеонаблюдения остаются системами сбора и хранения видеоданных.

Поэтому в работе предложены алгоритмы и методики работы с данными видеонаблюдения, содержащие изображения людей даже небольшого размера. К таким относятся данные, где размер изображения головы человека имеет размер от $24\times24$ пикселя.

% {\progress} 
% Этот раздел должен быть отдельным структурным элементом по
% ГОСТ, но он, как правило, включается в описание актуальности
% темы. Нужен он отдельным структурынм элемементом или нет ---
% смотрите другие диссертации вашего совета, скорее всего не нужен.

{\aim} данной работы является разработка методов повышения качества локализации, сопровождения и определения позы людей в видеопоследовательности, полученных статичной камерой, за счёт использования информации о калибровке камеры и движении людей в сцене.

Для~достижения поставленной цели необходимо было решить следующие {\tasks}:
\begin{enumerate}
	\item Разработать и реализовать алгоритм уточнения результатов обнаружения объектов интереса на изображении, использующий информацию о калибровке камеры, допускающий значения параметров калибровки из широкого диапазона.
	\item Разработать и реализовать алгоритм сопровождения людей в видеопоследовательности, превосходящий результаты существующих методов.
	\item Разработать модель позы человека в видео, содержащий модель движения человека в сцене, и эффективный алгоритм поиска оптимального значения разработанной модели.
\end{enumerate}

{\defpositions}
\begin{enumerate}
	\item Предложен и реализован новый алгоритм уточнения результатов обнаружения объектов интереса на изображении, использующий информацию о калибровке камеры, основанный на обучении по синтетическим прецедентам. Представлен способ интеграции предложенного алгоритма в существующий метод обнаружения голов людей, использующие подход скользящего окна. Эмпирически показана применимость предложенного метода для увеличения производительности и качества обнаружения для широкого диапазона значений параметров калибровки камеры.
	\item Предложен и реализован новый алгоритм сопровождения людей в видеопоследовательности, использующий устойчивый метод построения треклетов и информацию о регионах входа/выхода для заданной сцены. Превосходство предложенного метода перед аналогами показано путём сравнения результатов с эталоном.
	\item Предложено обобщение модели позы человека в видео моделью движения суставов в виде линейной динамической системы. Предложен и реализован алгоритм поиска локального оптимума представленной модели. Превосходство предложенной модели перед аналогами показано путём сравнения результатов с эталоном. 
\end{enumerate}

{\novelty}
\begin{enumerate}
  \item Впервые предложен способ построения отображения размеров и положений объектов на изображении в позу камеры, основанный на обучении по синтетическим прецедентам, позволяющий определять параметры позы камеры в широком диапазоне. Ранее существовавшие работы предлагали аналитические модели отображения и явно использовали предположение о близости угла наклона камеры к горизонтальному и/или отсутствию поворота камеры относительно оптической оси (крена камеры).
  \item Впервые предложен алгоритм уточнения результатов обнаружения объектов на изображении, использующий информацию о калибровке камеры, основанный на обучении по синтетическим прецедентам, и допускающий широкий диапазон значений параметров калибровки. Ранее существовавшие алгоритмы делали предположение о наклоне камеры близкому к горизонтальному.
  \item Впервые было предложены расширение модели позы человека в видео информацией о его движении в виде линейной динамической системы и эффективный алгоритм поиска локального оптимума предложенной модели. Ранее существовавшие модели рассматривали изменение позы человека между кадрами как допустимое отклонение от позы на соседних кадрах, то есть не учитывали скорость движения человека.
\end{enumerate}

{\influence} Предложенный алгоритм уточнения результатов обнаружения объектов интереса был применён к алгоритму обнаружения голов людей на изображении и интегрирован с ним. Эмпирическая оценка показала, что полученный алгоритм превосходит базовый одновременно и в скорости обработки данных, и в точности обнаружения. В работе представлен способ интеграции с произвольным алгоритмом обнаружения объектов, использующем подход скользящего окна. Полученный в результате алгоритм обнаружения требует задания параметров калибровки в виде фокусного расстояния и позы камеры в сцене. Однако, в работе также представлен метод определения позы камеры по результатам обнаружения объектов в сцене. Интеграция обоих предложенных алгоритмов с методом обнаружения объектов интереса позволяет уменьшить количество требуемых параметров только до фокусного расстояния оптической системы камеры. Дальнейшее объединение с существующими методами оценки фокусного расстояния позволит создать полностью автоматический алгоритм обнаружения объектов интереса, использующий информацию о калибровке камеры для обработки только релевантных регионов изображения, что повысит точность и скорость обработки кадров.

Результаты работы предложенных алгоритмов сопровождения и определения позы человека в видео превосходят результаты предшествующих методов, однако качество результатов анализа не позволяет их использование в коммерческих системах в автоматическом режиме. С другой стороны, предложенные алгоритмы используют генеративные модели построения результата, а следовательно допускают автоматизированный режим анализа видеоданных. Например, алгоритмы допускают построение решения, соответствующего частичной экспертной разметке. На основе этой идеи был предложен и реализован автоматизированный алгоритм построения эталонной выборки позы человека в видео, состоящий из двух повторяющихся шагов:
\begin{itemize}
	\item применение алгоритма поиска оптимальной позы человека в видео, соответствующего частичной экспертной разметке;
	\item расширение частичной экспертной разметки для исправления ошибок текущего решения.
\end{itemize}

%{\methods} \ldots

%{\reliability} полученных результатов обеспечивается \ldots \ Результаты находятся в соответствии с результатами, полученными другими авторами.


{\probation}
Основные результаты работы докладывались~на:
\begin{itemize}
	\item CMC MSU-Huawei International Workshop "Selected topics in multimedia image processing and analysis" (Москва, Россия, 31 августа 2016);
	\item 5th International Workshop on Image Mining. Theory and Applications (Берлин, Германия, 2015 год);
	\item 11th International Conference on Pattern Recognition and Image Analysis: New Information Technologies (Самара, Россия, 2013 год);
	\item 26-я Международная конференция по компьютерной графике, обработке изображений и машинному зрению, системам визуализации и виртуального окружения GraphiCon 2016 (Нижний Новгород, Россия, 19-23 сентября 2016 год);
	\item 25-я Международная конференция по компьютерной графике, обработке изображений и машинному зрению, системам визуализации и виртуального окружения GraphiCon 2015 (Протвино, Россия, 22-25 сентября 2015 год);	
	\item 24-я Международная конференция по компьютерной графике, обработке изображений и машинному зрению, системам визуализации и виртуального окружения GraphiCon 2014 (Ростов-на-Дону, Россия, 30 сентября-3 октября 2014 год);
\end{itemize}

{\contribution} Автор принимал активное участие \ldots

%\publications\ Основные результаты по теме диссертации изложены в ХХ печатных изданиях~\cite{Sokolov,Gaidaenko,Lermontov,Management},
%Х из которых изданы в журналах, рекомендованных ВАК~\cite{Sokolov,Gaidaenko}, 
%ХХ --- в тезисах докладов~\cite{Lermontov,Management}.

\ifnumequal{\value{bibliosel}}{0}{% Встроенная реализация с загрузкой файла через движок bibtex8
    \publications\ Основные результаты по теме диссертации изложены в XX печатных изданиях, 
    X из которых изданы в журналах, рекомендованных ВАК, 
    X "--- в тезисах докладов.%
}{% Реализация пакетом biblatex через движок biber
%Сделана отдельная секция, чтобы не отображались в списке цитированных материалов
    \begin{refsection}%
        \printbibliography[heading=countauthornotvak, env=countauthornotvak, keyword=biblioauthornotvak, section=1]%
        \printbibliography[heading=countauthorvak, env=countauthorvak, keyword=biblioauthorvak, section=1]%
        \printbibliography[heading=countauthorconf, env=countauthorconf, keyword=biblioauthorconf, section=1]%
        \printbibliography[heading=countauthor, env=countauthor, keyword=biblioauthor, section=1]%
        \publications\ Основные результаты по теме диссертации изложены в \arabic{citeauthor} печатных изданиях, 
        \arabic{citeauthorvak} из которых изданы в журналах, рекомендованных ВАК, 
        \nocite{konushin2017ispolzovanie48435059}
        \nocite{konushin2015an-improvement33378280}
        \nocite{konushin2015human9613323}
        \nocite{gringauz2016estimation34131922}
        \nocite{konushin2013improvement5112417}
        %\arabic{citeauthorconf} "--- в тезисах докладов.
        \nocite{gringauz2015an-algorithm10904785}
        \nocite{konushin2016further34136983}
        \nocite{gringauz2014modification7624365}
    \end{refsection}
} % Характеристика работы по структуре во введении и в автореферате не отличается (ГОСТ Р 7.0.11, пункты 5.3.1 и 9.2.1), потому её загружаем из одного и того же внешнего файла, предварительно задав форму выделения некоторым параметрам

%Диссертационная работа была выполнена при поддержке грантов ...

%\underline{\textbf{Объем и структура работы.}} Диссертация состоит из~введения, четырех глав, заключения и~приложения. Полный объем диссертации \textbf{ХХХ}~страниц текста с~\textbf{ХХ}~рисунками и~5~таблицами. Список литературы содержит \textbf{ХХX}~наименование.

%\newpage
\section*{Содержание работы}
Во \underline{\textbf{введении}} обосновывается актуальность исследований, проводимых в рамках данной диссертационной работы, формулируется цель, ставятся задачи работы, сформулированы научная новизна и практическая значимость представляемой работы.

\underline{\textbf{Первая глава}} посвящена обзору научной литературы по изучаемой проблеме.

\iffalse
 картинку можно добавить так:
\begin{figure}[ht] 
  \center
  \includegraphics [scale=0.27] {latex}
  \caption{Подпись к картинке.} 
  \label{img:latex}
\end{figure}

Формулы в строку без номера добавляются так:
\[ 
  \lambda_{T_s} = K_x\frac{d{x}}{d{T_s}}, \qquad
  \lambda_{q_s} = K_x\frac{d{x}}{d{q_s}},
\]

\fi

\underline{\textbf{Вторая глава}} посвящена исследованию задачи обнаружения людей на изображении в сценарии видеонаблюдения. Задача обнаружения объектов интереса на изображения "--- базовая задача компьютерного зрения. Классическим подходом к её решению является метод скользящего окна, основанный на классификации различных регионов изображения на несколько классов (в частном случае на два: <<объект интереса>>, <<другое>>).

В общем случае алгоритмы обнаружения не могут полагаться на априорное знание о положении объектов интереса в сцене и вынуждены просматривать огромное количество возможных положений и масштабов их изображений. Такой подход имеет два негативных последствия: 1) высокая вычислительная сложность анализа и 2) увеличение количества ложных срабатываний.

Сценарий видеонаблюдения имеет особенности, позволяющие получить такую априорную информацию о видеопоследовательности. Во-первых, большинство анализируемых сцен состоят из единственной плоскости земли, где могут находиться объекты интереса (люди или автомобили). Во-вторых, камеры видеонаблюдения являются статичными и параметры их калибровки не меняются со временем. Эти предположения порождают две задачи: 1) определение параметров калибровки камеры по результатам анализа сцены и 2) интеграция параметров калибровки с алгоритмами обнаружения для уточнения результатов обнаружения.

Параметры калибровки камеры разделяют на две части: параметры внешней калибровки, определяющие положение и ориентацию камеры в сцене, и параметры внутренней калибровки, определяющие свойства камеры. Параметры внешней калибровки также называют её позой.

Существующие подходы определения параметров позы камеры и применение её калибровки для уточнения результатов обнаружения накладывают жесткие ограничения на возможные значения позы камеры. Среди таких ограничений представлены отсутствие крена камеры или близость угла наклона к нулю. Причинами таких ограничений являются аналитические модели используемые при построении отображений между размерами и положениями объектов в сцене и параметрами калибровки камеры.

Входными параметрами алгоритма определения позы камеры является видеопоследовательность, снятая статичной камерой, значение её фокусного расстояния и алгоритм локализации людей в сцене. Отличительной особенностью предложенного метода является использование синтетической выборки для построения искомого отображения. Описанные выше предположения о наблюдаемой сцене позволяют построить генеративную математическую модель наблюдаемых данных, состоящую из:
\begin{itemize}
	\item модели сцены в виде плоскости $z=0$;
	\item модели камеры, описываемой фокусным расстоянием $f$;
	\item модели человека \cite{pishchulin15arxiv}.
\end{itemize}
В предложенной модели наблюдаемой сцены поза камеры описывается тремя параметрами: высотой над плоскостью земли $h$, углом наклона $t$ и углом крена $r$. Предложенная модель является генеративной и позволяет синтезировать изображения с помощью средств компьютерной графики. Была построена синтетическая выборка данных видеонаблюдения, состоящая из 100374 различных сцен, каждая из которых характеризуется параметрами калибровки камеры в наблюдаемой сцене. В каждой сцене находилось не менее 200 человек, равномерно распределенных по изображению.

Домены реальных данных видеонаблюдения и синтезированных из модели существенно отличаются из-за отсутствия модели освещения, текстуры и вариативности позы людей. Поэтому в работы было необходимо выбрать признаковое описание инвариантное к такому изменению домена. В качестве признаково описания было предложено использовать результаты обнаружения людей на изображении. Алгоритм обнаружения голов людей на изображении \cite{prisacariu_reid_tr2310_09} был выбран для построения признаково описания. Таким образом, каждый человек в сцене описывается положением центра и размерами ограничивающего прямоугольник его головы.

Для построения отображения признаково описания сцены и фокусного расстояния в значения параметров позы камеры было предложено использовать нейронную сеть. Одному и тому же положению и размерам головы человека в сцене могут соответствовать различные параметры позы камеры. Поэтому регрессия позы камеры осуществляется по прецеденту, объединяющему 64 равномерно распределенных обнаружения человека в сцене. Чтобы порядок описания людей в сцене не влиял на результаты регрессии, люди в прецеденте упорядочены по размеру головы человека.

Синтетическая выборка не содержит результатов ложных срабатываний обнаружения людей в сцене, однако они часто встречаются на практике. Поэтому в обучающие прецеденты были добавлены шумы в виде ограничивающих прямоугольников со случайно выбранными положениями и размерами и дублирующиеся обнаружения объектов.

Таким образом, входными параметрами нейронной сети являлся прецедент $x$, состоящий из 64 положений голов людей в сцене и значений фокусного расстояния камеры. Предсказание нейронной сети интерпретировалось как математическое ожидание $\tilde{l}_c(x, \Theta)$ и параметры матрицы ковариации $\Sigma_c(x, \Theta)$ распределения позы камеры, пораждающей наблюдаемые данные. В работе использовалось предположение, что данное распределение является нормальным:
\begin{equation}
	p(l_{c}|x, \Theta) = N(l_c|\tilde{l_c}(x, \Theta), \Sigma_c(x, \Theta))
\end{equation}
Обучение производилось путем минимизации отрицательного логарифма функции правдоподобия.

Предсказание параметров матрицы ковариации распределения позы камеры позволяет решить несколько задач: 1) уменьшить влияние сложных примеров на процедуру обучения и 2) оценить уверенность регрессора в предсказании на этапе тестирования. Сложными для определения позы камеры являются прецеденты, содержащие большое количество ложных или повторяющихся срабатываний алгоритма обнаружения голов. Значение определителя матрицы ковариации является оценкой степени неуверенности в построенном предсказании.

Тестирование алгоритма проводилось на синтетических данных и на открытых выборках видеонаблюдения. Результаты тестирования показали, что на реальных и синтетических данных, не содержащих ложных срабатываний алгоритма детектирования, поза камеры определятся верно с точностью в 3 отклонения маргинального распределения параметров. Также тестирование показало увеличение значения предсказанной дисперсии маргинальных отклонений параметров позы при увеличении количества шума или уменьшении разнообразия обнаружений в прецеденте.

Уточнение результатов обнаружения людей в сценарии видеонаблюдения по информации о калибровке камеры является обратной к предыдущей рассматриваемой задаче.

\underline{\textbf{Третья глава}} посвящена исследованию задачи сопровождения людей в видеопоследовательности.

В \underline{\textbf{четвертой главе}} приведено описание предложенной модели позы человека в видео, содержащей модель движения суставов. Также предложены алгоритмы поиска локального оптимума предложенной модели.

В \underline{\textbf{заключении}} приведены основные результаты работы, которые заключаются в следующем:
\input{common/concl}

\iffalse
%\newpage
При использовании пакета \verb!biblatex! список публикаций автора по теме
диссертации формируется в разделе <<\publications>>\ файла
\verb!../common/characteristic.tex!  при помощи команды \verb!\nocite! 
\fi

\ifdefmacro{\microtypesetup}{\microtypesetup{protrusion=false}}{} % не рекомендуется применять пакет микротипографики к автоматически генерируемому списку литературы
\ifnumequal{\value{bibliosel}}{0}{% Встроенная реализация с загрузкой файла через движок bibtex8
  \renewcommand{\refname}{\large \authorbibtitle}
  \nocite{*}
  \insertbiblioauthor                          % Подключаем Bib-базы
  %\insertbiblioother   % !!! bibtex не умеет работать с несколькими библиографиями !!!
}{% Реализация пакетом biblatex через движок biber
  \insertbiblioauthor                          % Подключаем Bib-базы
  %\insertbiblioauthorgrouped
  \insertbiblioother
}
\ifdefmacro{\microtypesetup}{\microtypesetup{protrusion=true}}{}

